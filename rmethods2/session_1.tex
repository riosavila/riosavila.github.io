% Options for packages loaded elsewhere
\PassOptionsToPackage{unicode}{hyperref}
\PassOptionsToPackage{hyphens}{url}
\PassOptionsToPackage{dvipsnames,svgnames,x11names}{xcolor}
%
\documentclass[
  letterpaper,
  DIV=11,
  numbers=noendperiod]{scrartcl}

\usepackage{amsmath,amssymb}
\usepackage{iftex}
\ifPDFTeX
  \usepackage[T1]{fontenc}
  \usepackage[utf8]{inputenc}
  \usepackage{textcomp} % provide euro and other symbols
\else % if luatex or xetex
  \usepackage{unicode-math}
  \defaultfontfeatures{Scale=MatchLowercase}
  \defaultfontfeatures[\rmfamily]{Ligatures=TeX,Scale=1}
\fi
\usepackage{lmodern}
\ifPDFTeX\else  
    % xetex/luatex font selection
\fi
% Use upquote if available, for straight quotes in verbatim environments
\IfFileExists{upquote.sty}{\usepackage{upquote}}{}
\IfFileExists{microtype.sty}{% use microtype if available
  \usepackage[]{microtype}
  \UseMicrotypeSet[protrusion]{basicmath} % disable protrusion for tt fonts
}{}
\makeatletter
\@ifundefined{KOMAClassName}{% if non-KOMA class
  \IfFileExists{parskip.sty}{%
    \usepackage{parskip}
  }{% else
    \setlength{\parindent}{0pt}
    \setlength{\parskip}{6pt plus 2pt minus 1pt}}
}{% if KOMA class
  \KOMAoptions{parskip=half}}
\makeatother
\usepackage{xcolor}
\setlength{\emergencystretch}{3em} % prevent overfull lines
\setcounter{secnumdepth}{-\maxdimen} % remove section numbering
% Make \paragraph and \subparagraph free-standing
\ifx\paragraph\undefined\else
  \let\oldparagraph\paragraph
  \renewcommand{\paragraph}[1]{\oldparagraph{#1}\mbox{}}
\fi
\ifx\subparagraph\undefined\else
  \let\oldsubparagraph\subparagraph
  \renewcommand{\subparagraph}[1]{\oldsubparagraph{#1}\mbox{}}
\fi

\usepackage{color}
\usepackage{fancyvrb}
\newcommand{\VerbBar}{|}
\newcommand{\VERB}{\Verb[commandchars=\\\{\}]}
\DefineVerbatimEnvironment{Highlighting}{Verbatim}{commandchars=\\\{\}}
% Add ',fontsize=\small' for more characters per line
\usepackage{framed}
\definecolor{shadecolor}{RGB}{241,243,245}
\newenvironment{Shaded}{\begin{snugshade}}{\end{snugshade}}
\newcommand{\AlertTok}[1]{\textcolor[rgb]{0.68,0.00,0.00}{#1}}
\newcommand{\AnnotationTok}[1]{\textcolor[rgb]{0.37,0.37,0.37}{#1}}
\newcommand{\AttributeTok}[1]{\textcolor[rgb]{0.40,0.45,0.13}{#1}}
\newcommand{\BaseNTok}[1]{\textcolor[rgb]{0.68,0.00,0.00}{#1}}
\newcommand{\BuiltInTok}[1]{\textcolor[rgb]{0.00,0.23,0.31}{#1}}
\newcommand{\CharTok}[1]{\textcolor[rgb]{0.13,0.47,0.30}{#1}}
\newcommand{\CommentTok}[1]{\textcolor[rgb]{0.37,0.37,0.37}{#1}}
\newcommand{\CommentVarTok}[1]{\textcolor[rgb]{0.37,0.37,0.37}{\textit{#1}}}
\newcommand{\ConstantTok}[1]{\textcolor[rgb]{0.56,0.35,0.01}{#1}}
\newcommand{\ControlFlowTok}[1]{\textcolor[rgb]{0.00,0.23,0.31}{#1}}
\newcommand{\DataTypeTok}[1]{\textcolor[rgb]{0.68,0.00,0.00}{#1}}
\newcommand{\DecValTok}[1]{\textcolor[rgb]{0.68,0.00,0.00}{#1}}
\newcommand{\DocumentationTok}[1]{\textcolor[rgb]{0.37,0.37,0.37}{\textit{#1}}}
\newcommand{\ErrorTok}[1]{\textcolor[rgb]{0.68,0.00,0.00}{#1}}
\newcommand{\ExtensionTok}[1]{\textcolor[rgb]{0.00,0.23,0.31}{#1}}
\newcommand{\FloatTok}[1]{\textcolor[rgb]{0.68,0.00,0.00}{#1}}
\newcommand{\FunctionTok}[1]{\textcolor[rgb]{0.28,0.35,0.67}{#1}}
\newcommand{\ImportTok}[1]{\textcolor[rgb]{0.00,0.46,0.62}{#1}}
\newcommand{\InformationTok}[1]{\textcolor[rgb]{0.37,0.37,0.37}{#1}}
\newcommand{\KeywordTok}[1]{\textcolor[rgb]{0.00,0.23,0.31}{#1}}
\newcommand{\NormalTok}[1]{\textcolor[rgb]{0.00,0.23,0.31}{#1}}
\newcommand{\OperatorTok}[1]{\textcolor[rgb]{0.37,0.37,0.37}{#1}}
\newcommand{\OtherTok}[1]{\textcolor[rgb]{0.00,0.23,0.31}{#1}}
\newcommand{\PreprocessorTok}[1]{\textcolor[rgb]{0.68,0.00,0.00}{#1}}
\newcommand{\RegionMarkerTok}[1]{\textcolor[rgb]{0.00,0.23,0.31}{#1}}
\newcommand{\SpecialCharTok}[1]{\textcolor[rgb]{0.37,0.37,0.37}{#1}}
\newcommand{\SpecialStringTok}[1]{\textcolor[rgb]{0.13,0.47,0.30}{#1}}
\newcommand{\StringTok}[1]{\textcolor[rgb]{0.13,0.47,0.30}{#1}}
\newcommand{\VariableTok}[1]{\textcolor[rgb]{0.07,0.07,0.07}{#1}}
\newcommand{\VerbatimStringTok}[1]{\textcolor[rgb]{0.13,0.47,0.30}{#1}}
\newcommand{\WarningTok}[1]{\textcolor[rgb]{0.37,0.37,0.37}{\textit{#1}}}

\providecommand{\tightlist}{%
  \setlength{\itemsep}{0pt}\setlength{\parskip}{0pt}}\usepackage{longtable,booktabs,array}
\usepackage{calc} % for calculating minipage widths
% Correct order of tables after \paragraph or \subparagraph
\usepackage{etoolbox}
\makeatletter
\patchcmd\longtable{\par}{\if@noskipsec\mbox{}\fi\par}{}{}
\makeatother
% Allow footnotes in longtable head/foot
\IfFileExists{footnotehyper.sty}{\usepackage{footnotehyper}}{\usepackage{footnote}}
\makesavenoteenv{longtable}
\usepackage{graphicx}
\makeatletter
\def\maxwidth{\ifdim\Gin@nat@width>\linewidth\linewidth\else\Gin@nat@width\fi}
\def\maxheight{\ifdim\Gin@nat@height>\textheight\textheight\else\Gin@nat@height\fi}
\makeatother
% Scale images if necessary, so that they will not overflow the page
% margins by default, and it is still possible to overwrite the defaults
% using explicit options in \includegraphics[width, height, ...]{}
\setkeys{Gin}{width=\maxwidth,height=\maxheight,keepaspectratio}
% Set default figure placement to htbp
\makeatletter
\def\fps@figure{htbp}
\makeatother

\KOMAoption{captions}{tableheading}
\makeatletter
\@ifpackageloaded{caption}{}{\usepackage{caption}}
\AtBeginDocument{%
\ifdefined\contentsname
  \renewcommand*\contentsname{Table of contents}
\else
  \newcommand\contentsname{Table of contents}
\fi
\ifdefined\listfigurename
  \renewcommand*\listfigurename{List of Figures}
\else
  \newcommand\listfigurename{List of Figures}
\fi
\ifdefined\listtablename
  \renewcommand*\listtablename{List of Tables}
\else
  \newcommand\listtablename{List of Tables}
\fi
\ifdefined\figurename
  \renewcommand*\figurename{Figure}
\else
  \newcommand\figurename{Figure}
\fi
\ifdefined\tablename
  \renewcommand*\tablename{Table}
\else
  \newcommand\tablename{Table}
\fi
}
\@ifpackageloaded{float}{}{\usepackage{float}}
\floatstyle{ruled}
\@ifundefined{c@chapter}{\newfloat{codelisting}{h}{lop}}{\newfloat{codelisting}{h}{lop}[chapter]}
\floatname{codelisting}{Listing}
\newcommand*\listoflistings{\listof{codelisting}{List of Listings}}
\makeatother
\makeatletter
\makeatother
\makeatletter
\@ifpackageloaded{caption}{}{\usepackage{caption}}
\@ifpackageloaded{subcaption}{}{\usepackage{subcaption}}
\makeatother
\ifLuaTeX
  \usepackage{selnolig}  % disable illegal ligatures
\fi
\IfFileExists{bookmark.sty}{\usepackage{bookmark}}{\usepackage{hyperref}}
\IfFileExists{xurl.sty}{\usepackage{xurl}}{} % add URL line breaks if available
\urlstyle{same} % disable monospaced font for URLs
\hypersetup{
  pdftitle={Research Methods II},
  pdfauthor={Fernando Rios-Avila},
  colorlinks=true,
  linkcolor={blue},
  filecolor={Maroon},
  citecolor={Blue},
  urlcolor={Blue},
  pdfcreator={LaTeX via pandoc}}

\title{Research Methods II}
\usepackage{etoolbox}
\makeatletter
\providecommand{\subtitle}[1]{% add subtitle to \maketitle
  \apptocmd{\@title}{\par {\large #1 \par}}{}{}
}
\makeatother
\subtitle{Session 1: Surveys, IO and SAM}
\author{Fernando Rios-Avila}
\date{}

\begin{document}
\maketitle
\subsection{Why are we here?}\label{why-are-we-here}

\begin{itemize}
\item
  If you are reading this, you are probably interested in the
  Microeconometrics path of Research methods II.
\item
  This course assumes you are familiar with the basics of econometrics
  and statistics.

  \begin{itemize}
  \tightlist
  \item
    We will not cover questions from Research methods I. You probably
    know the answers to those questions already!
  \end{itemize}
\item
  This course will focus on tools that are commonly used in empirical
  research in economics:

  \begin{itemize}
  \item
    We will emphasize the use of Household Surveys (with the issues it
    entails)
  \item
    And focus on applied econometrics using cross-sectional data, for
    distributional analysis and policy simulation.
  \end{itemize}
\item
  Thus, we will do much more use of empirical tools and software than we
  did in Research methods I.
\item
  We will have 7 Sessions, 2 homeworks and a final project.
\end{itemize}

\section{Surveys: What are they?}\label{surveys-what-are-they}

\subsection{What is a Survey?}\label{what-is-a-survey}

\begin{itemize}
\tightlist
\item
  A Survey is a source of data that aims to collect information from a
  population of interest, to underestand some characteristics,
  behaviors, or opinions of that population as a whole.

  \begin{itemize}
  \tightlist
  \item
    The population of interest can be individuals, households, firms,
    etc.
  \end{itemize}
\item
  They can be useful to identify and analyze policy questions.
\item
  However, they are secondary data, and thus have limitations in terms
  of the questions that can be answered with them.

  \begin{itemize}
  \tightlist
  \item
    You cannot answer questions that require data that was not
    collected.
  \item
    They can also be limited to Interivewee recall, or willingness to
    answer.
  \item
    Or how accessible the population of interest is.
  \end{itemize}
\end{itemize}

\subsection{Example of Surveys}\label{example-of-surveys}

\begin{itemize}
\tightlist
\item
  Current Population Survey (CPS)

  \begin{itemize}
  \tightlist
  \item
    Monthly survey of 60,000 households in the US.
  \end{itemize}
\item
  American Community Survey (ACS)

  \begin{itemize}
  \tightlist
  \item
    Annual survey of 3.5 million households in the US.
  \end{itemize}
\item
  American Time Use Survey (ATUS)

  \begin{itemize}
  \tightlist
  \item
    Annual survey of 6,000 individuals in the US.
  \end{itemize}
\item
  Enterprise Surveys - WB (ES)

  \begin{itemize}
  \tightlist
  \item
    Survey of firms in developing countries. Different years of
    Collection
  \end{itemize}
\end{itemize}

\subsection{What makes a good survey?}\label{what-makes-a-good-survey}

\begin{itemize}
\item
  A Good survey is one that allows you to obtain estimates of statistics
  of interest for the population with ``Tolaraable'' levels of Accuracy.
\item
  To do this, you need to have a good sampling design (representation
  and ``independence of the population'') and a good questionnaire
  design (questions that are clear and easy to answer).
\item
  A good survey needs to be representative of the population of
  interest. To do this appropriately, data will be collected based on a
  \textbf{frame} that will be used to select the sample.
\end{itemize}

\subsection{Types of Data Seletion}\label{types-of-data-seletion}

\subsection{Simple}

\begin{itemize}
\item
  Each observation in the ``frame'' has the same probability of being
  selected in the sample.
\item
  It may be difficult to implement in practice, because of cost and
  logistics. (distance)
\item
  It could also have problems of representativeness for small groups.
  (rare events)
\end{itemize}

\subsection{Clustered}

\begin{itemize}
\tightlist
\item
  Using some criteria, location for example, the population is divided
  into clusters.
\item
  For the sample selection, certain clusters are selected at random, and
  ``some'' observations within each selected.
\item
  This is more feasible in practice, because takes advantage of the
  ``clustering'' of the population.
\item
  However, one may need to account for possible ``common shocks'' that
  people within the same cluster may have.
\end{itemize}

\subsection{Stratified}

\begin{itemize}
\tightlist
\item
  Some times, statistics are required to accurately represent certain
  groups of the population. (by region, race, income level, etc)
\item
  In such cases, data can be collected in a way that ensures that the
  sample has enough observations for each group of interest.

  \begin{itemize}
  \tightlist
  \item
    It would be as collecting multiple samples, one for each group of
    interest.
  \end{itemize}
\item
  Within each Strata, it would also be possible to use a simple or
  clustered sampling design.
\end{itemize}

\subsection{What to be aware of?}\label{what-to-be-aware-of}

\begin{itemize}
\tightlist
\item
  The sampling design is important to ensure that the sample is
  representative of the population of interest.
\item
  However, there are limitations:

  \begin{itemize}
  \tightlist
  \item
    Not every-one selected will respond to the survey. (Is it random? )
  \item
    Rarey one assumes equal probability of selection. (Different
    sampling weights)
  \item
    Use of Stratatification and Clustering may require special
    treatment.
  \end{itemize}
\item
  Something else: Panel data

  \begin{itemize}
  \tightlist
  \item
    Because of attrition, it can be difficult to analyze if
    representativeity is required.
  \item
    However, it can be useful to analyze dynamics.
  \end{itemize}
\end{itemize}

\subsection{Descriptive Statistics}\label{descriptive-statistics}

\begin{itemize}
\item
  Once you have your data, you can start analyzing it by simply applying
  your survey weights.

  \begin{itemize}
  \tightlist
  \item
    Point estimates are straightforward to obtain.
  \end{itemize}
\item
  However, when considering the estimation of precision of the estimates
  (Variance and Standard Errors), there are two approaches that are
  important to consider:
\item
  \textbf{Finite Population Approach}:

  \begin{itemize}
  \tightlist
  \item
    Associated with data description
  \item
    Assumes that the population is finite, and thus Selection
    probabilities are not independent.
  \end{itemize}
\item
  \textbf{Superpopulation Approach}:

  \begin{itemize}
  \tightlist
  \item
    Associated with data modeling.
  \item
    Population is infinite. Selection probabilties are independent.
  \end{itemize}
\item
  For practical purposes, the difference between the two approaches is
  not that important.

  \begin{itemize}
  \tightlist
  \item
    With large enough samples, he ``finite sample correction'' is
    negligible.
  \end{itemize}
\end{itemize}

\subsection{Basic Summary Statistics:}\label{basic-summary-statistics}

\begin{itemize}
\tightlist
\item
  \(n\) is sample size. \(N\) is population size.
\item
  \(a_i\) an indicator of belonging to the sample. \(\sum a_i=n\)
\item
  Assume \(y_i\) is the outcome of interest, Say income.
  \[\texttt{mean: }\hat{\mu}_y=\bar{y} = \frac{1}{n}\sum_{i=1}^N a_i y_i = \frac{1}{n}\sum_{j=1}^n y_j\]
  \[\texttt{Variance: }\hat \sigma^2_y = \frac{1}{n-1} \sum_{i=1}^N a_i (y_i - \hat{\bar{y}} )^2\]
  \[\texttt{V of mean: }\widehat{V}(\bar{y}) = \frac{\sigma^2_y }{n} * fpc\]
\end{itemize}

Where \(fpc = \frac{N-n}{N-1}\) is the finite population correction.

\subsection{Accounting for Weights}\label{accounting-for-weights}

\begin{itemize}
\item
  The previous formulas did not account for weights.
\item
  Weights are factors used to ``reweight/expand'' the sample to make it
  representative of the population.
\item
  They can are typically related to the inverse of the probability of
  selection.
\item
  Simply said, a weight \(w_i=\frac{1}{n*\pi_i}\) is a measure of how
  many observations in the population are represented by observation
  \(i\) in the sample.
\end{itemize}

But how do we account for weights in the formulas?

\subsection{Summary Statistics with
Weights}\label{summary-statistics-with-weights}

Population: \[\hat N = \sum_{i=1}^n w_i\]

Normalized weights
\[v_i = \frac{n w_i}{ \sum w_i} \rightarrow E(v_i) = 1\]

Mean:
\[\hat{\mu}_y=\bar{y} = \frac{1}{\hat N}\sum_{i=1}^n w_i y_i = \frac{1}{n}\sum v_i y_i\]

\subsection{Variance with Weights}\label{variance-with-weights}

Variances are a bit more complicated. Normally you would consider:

\[Var(\bar y) = \frac{1}{n} \hat \sigma^2_y =\frac{1}{n} \frac{1}{n-1} \sum_{i=1}^n v_i (y_i - \bar y)^2\]

However, with survey weights you need to consider something else:
\[Var(\bar y) = \frac{1}{n} \sum_{i=1}^n v_i^2 (y_i - \bar y)^2\]

Which is similar to ``robust'' Standard errors in OLS.

\begin{quote}
What about Clusters and Strata?
\end{quote}

\subsection{How to account for weights in
Stata?}\label{how-to-account-for-weights-in-stata}

Lets use data from the National Health and Nutrition Examination Survey
(NHANES) to illustrate how to account for weights in Stata.

\begin{Shaded}
\begin{Highlighting}[]
\NormalTok{ frause oaxaca, }\KeywordTok{clear}
\NormalTok{** Create Weights \textless{}{-} This will }\KeywordTok{be}\NormalTok{ provided}
\KeywordTok{sum}\NormalTok{ wt, }\KeywordTok{meanonly}
\KeywordTok{replace}\NormalTok{ wt = }\FunctionTok{round}\NormalTok{(wt/}\FunctionTok{r}\NormalTok{(}\FunctionTok{min}\NormalTok{))}
\KeywordTok{gen}\NormalTok{ wage = }\FunctionTok{exp}\NormalTok{(lnwage)}
\end{Highlighting}
\end{Shaded}

\begin{verbatim}
<IPython.core.display.HTML object>
\end{verbatim}

\begin{verbatim}
(Excerpt from the Swiss Labor Market Survey 1998)
(1,647 real changes made)
(213 missing values generated)
\end{verbatim}

Summary Statistics:

\begin{Shaded}
\begin{Highlighting}[]
\NormalTok{** unweighted}
\KeywordTok{sum}\NormalTok{ wage,}\KeywordTok{d}
\NormalTok{** weighted}
\KeywordTok{sum}\NormalTok{ wage [aw=wt],}\KeywordTok{d}
\end{Highlighting}
\end{Shaded}

\begin{verbatim}

                            wage
-------------------------------------------------------------
      Percentiles      Smallest
 1%     3.907204       1.661434
 5%      11.8007       1.873127
10%      17.4216       2.197802       Obs               1,434
25%     23.19902       2.442003       Sum of wgt.       1,434

50%     30.08896                      Mean           32.39167
                        Largest       Std. dev.      16.12498
75%      38.5662       137.3627
90%     49.95005       152.6252       Variance       260.0151
95%     58.71271       164.8352       Skewness       2.486954
99%     85.47008       192.3077       Kurtosis       18.23702

                            wage
-------------------------------------------------------------
      Percentiles      Smallest
 1%     3.453689       1.661434
 5%     7.370678       1.873127
10%     15.83933       2.197802       Obs               1,434
25%     21.72247       2.442003       Sum of wgt.       2,686

50%     29.48271                      Mean            31.5322
                        Largest       Std. dev.      16.32811
75%     38.46154       137.3627
90%     49.95005       152.6252       Variance       266.6071
95%     58.11497       164.8352       Skewness       2.081806
99%     85.47008       192.3077       Kurtosis       14.68647
\end{verbatim}

Accounting for weights for summary Statistics

\begin{Shaded}
\begin{Highlighting}[]
\NormalTok{** unweighted}
\KeywordTok{mean}\NormalTok{ wage }
\NormalTok{** weighted}
\KeywordTok{mean}\NormalTok{ wage [pw=wt]}
\KeywordTok{mean}\NormalTok{ wage [pw=wt], }\BaseNTok{over}\NormalTok{(female)}
\end{Highlighting}
\end{Shaded}

\begin{verbatim}

Mean estimation                          Number of obs = 1,434

--------------------------------------------------------------
             |       Mean   Std. err.     [95% conf. interval]
-------------+------------------------------------------------
        wage |   32.39167   .4258186      31.55637    33.22696
--------------------------------------------------------------

Mean estimation                          Number of obs = 1,434

--------------------------------------------------------------
             |       Mean   Std. err.     [95% conf. interval]
-------------+------------------------------------------------
        wage |    31.5322   .4765835      30.59733    32.46708
--------------------------------------------------------------

Mean estimation                           Number of obs = 1,434

---------------------------------------------------------------
              |       Mean   Std. err.     [95% conf. interval]
--------------+------------------------------------------------
c.wage@female |
           0  |   33.87649   .6152059      32.66969    35.08329
           1  |   28.76133    .722173       27.3447    30.17796
---------------------------------------------------------------
\end{verbatim}

Tables and cross tables

\begin{Shaded}
\begin{Highlighting}[]
\NormalTok{** unweighted}
\KeywordTok{tab}\NormalTok{ educ female}

\KeywordTok{tab}\NormalTok{ educ female [}\FunctionTok{w}\NormalTok{=wt]}
\end{Highlighting}
\end{Shaded}

\begin{verbatim}

           |   sex of respondent
  years of |      (1=female)
 education |         0          1 |     Total
-----------+----------------------+----------
         5 |         6         23 |        29 
         9 |        47         93 |       140 
      9.75 |         8         26 |        34 
        10 |        10         42 |        52 
      10.5 |       376        447 |       823 
      11.5 |         7         14 |        21 
        12 |       111         87 |       198 
      12.5 |        56         80 |       136 
        15 |        60         13 |        73 
      17.5 |        78         63 |       141 
-----------+----------------------+----------
     Total |       759        888 |     1,647 

           |   sex of respondent
  years of |      (1=female)
 education |         0          1 |     Total
-----------+----------------------+----------
         5 |        17         45 |        62 
         9 |       104        181 |       285 
      9.75 |        14         52 |        66 
        10 |        22         80 |       102 
      10.5 |       725        833 |     1,558 
      11.5 |        19         27 |        46 
        12 |       201        148 |       349 
      12.5 |       108        157 |       265 
        15 |       111         21 |       132 
      17.5 |       149        111 |       260 
-----------+----------------------+----------
     Total |     1,470      1,655 |     3,125 
\end{verbatim}

Better approach: \texttt{svyset}

\begin{Shaded}
\begin{Highlighting}[]
\KeywordTok{webuse}\NormalTok{ nhanes2f, }\KeywordTok{clear}
\KeywordTok{svyset}\NormalTok{ psuid }\CommentTok{/// Cluster}
\NormalTok{       [}\KeywordTok{pweight}\NormalTok{=finalwgt], }\CommentTok{/// Survey Weight as Inverse of Prob of Selection}
       \KeywordTok{strata}\NormalTok{(stratid)     }\CommentTok{// Strata Identifier}
\KeywordTok{mean}\NormalTok{ zinc      }
\KeywordTok{mean}\NormalTok{ zinc [pw=finalwgt]}
\KeywordTok{svy}\NormalTok{: }\KeywordTok{mean}\NormalTok{ zinc      }
\end{Highlighting}
\end{Shaded}

\begin{verbatim}

Sampling weights: finalwgt
             VCE: linearized
     Single unit: missing
        Strata 1: stratid
 Sampling unit 1: psuid
           FPC 1: <zero>

Mean estimation                          Number of obs = 9,189

--------------------------------------------------------------
             |       Mean   Std. err.     [95% conf. interval]
-------------+------------------------------------------------
        zinc |   86.51518   .1510744      86.21904    86.81132
--------------------------------------------------------------

Mean estimation                          Number of obs = 9,189

--------------------------------------------------------------
             |       Mean   Std. err.     [95% conf. interval]
-------------+------------------------------------------------
        zinc |   87.18207   .1828747      86.82359    87.54054
--------------------------------------------------------------
(running mean on estimation sample)

Survey: Mean estimation

Number of strata = 31            Number of obs   =       9,189
Number of PSUs   = 62            Population size = 104,176,071
                                 Design df       =          31

--------------------------------------------------------------
             |             Linearized
             |       Mean   std. err.     [95% conf. interval]
-------------+------------------------------------------------
        zinc |   87.18207   .4944827      86.17356    88.19057
--------------------------------------------------------------
\end{verbatim}

\subsection{Testing Significance across 2
groups}\label{testing-significance-across-2-groups}

Consider two groups with the following characteristics:

\begin{longtable}[]{@{}llll@{}}
\toprule\noalign{}
Group & N & mean & Var \\
\midrule\noalign{}
\endhead
\bottomrule\noalign{}
\endlastfoot
1 & 100 & 45 & 32.56 \\
2 & 150 & 55 & 21.97 \\
\end{longtable}

\begin{itemize}
\item
  Is the difference in means statistically significant?
\item
  Test \(H_0: \mu_1 = \mu_2\) vs \(H_1: \mu_1 \neq \mu_2\)
\end{itemize}

\[t = \frac{\mu_2 - \mu_1}{\sqrt{\frac{\sigma_1^2}{n_1} + \frac{\sigma_2^2}{n_2}}}
=\frac{10}{\sqrt{\frac{32.56}{100}+\frac{21.97}{150}}}=\frac{10}{.47207} = 14.55453
\]

\begin{itemize}
\item
  If \(t\) is large enough, we can reject the null hypothesis.
\item
  But this does not work if you have weights\ldots{}
\end{itemize}

\subsection{Testing Significance across 2
groups}\label{testing-significance-across-2-groups-1}

\begin{itemize}
\tightlist
\item
  But you can use OLS to test the difference in means with weights!

  \begin{itemize}
  \tightlist
  \item
    Make sure you use ``robust'' standard errors, or ``pw'' option.
  \end{itemize}
\end{itemize}

\begin{Shaded}
\begin{Highlighting}[]
\NormalTok{ frause oaxaca, }\KeywordTok{clear}
\NormalTok{** Create Weights \textless{}{-} This will }\KeywordTok{be}\NormalTok{ provided}
\KeywordTok{sum}\NormalTok{ wt, }\KeywordTok{meanonly}
\KeywordTok{replace}\NormalTok{ wt = }\FunctionTok{round}\NormalTok{(wt/}\FunctionTok{r}\NormalTok{(}\FunctionTok{min}\NormalTok{))}
\KeywordTok{gen}\NormalTok{ wage = }\FunctionTok{exp}\NormalTok{(lnwage)}
\KeywordTok{reg}\NormalTok{ wage i.female [pw=wt],}
\end{Highlighting}
\end{Shaded}

\begin{verbatim}
(Excerpt from the Swiss Labor Market Survey 1998)
(1,647 real changes made)
(213 missing values generated)
(sum of wgt is 2,686)

Linear regression                               Number of obs     =      1,434
                                                F(1, 1432)        =      29.05
                                                Prob > F          =     0.0000
                                                R-squared         =     0.0244
                                                Root MSE          =     16.133

------------------------------------------------------------------------------
             |               Robust
        wage | Coefficient  std. err.      t    P>|t|     [95% conf. interval]
-------------+----------------------------------------------------------------
    1.female |  -5.115156   .9490209    -5.39   0.000    -6.976776   -3.253536
       _cons |   33.87649   .6154206    55.05   0.000     32.66927    35.08371
------------------------------------------------------------------------------
\end{verbatim}

\begin{itemize}
\tightlist
\item
  you can also use \texttt{svy:\ regress} for complex designs.
\end{itemize}

see
\href{https://stats.oarc.ucla.edu/stata/faq/how-can-i-do-a-t-test-with-survey-data/}{here}
for additional examples

\section{IO-Tables}\label{io-tables}

\subsection{IO-Tables}\label{io-tables-1}

\begin{itemize}
\item
  IO tables stand for Input-Output tables. They are a way to represent
  the production structure of an economy.
\item
  They provide a Static representation of the Economy
\item
  Each Row represents the production of a sector, and each column
  represents the use of that production by other sectors as Inputs.
\item
  The information it contains represent a snapshot of the economy at a
  given point in time.
\item
  It can be used to simulate changes in production, labor demand, and
  total production in the economy, under specific assumptions
  (production function)
\end{itemize}

\subsection{}\label{section}

\begin{itemize}
\item
  Consider a Economy with K=3 sectors: Agriculture, Manufacturing and
  Services, with a Final consumer agent (households)
\item
  Each sector (i) produces a good (\(X_i\)), which is sold to other
  sectors or the final consumer.
\item
  \(X_{ij}\) is the quantity of goods sector \(i\) sells to sector
  \(j\), and \(y_i\) the final consumption by households.
\item
  \(X_{ji}\) is also the quantity of goods sector \(i\) uses from sector
  \(j\) as inputs.
\item
  \(L_i\) is the amount of labor used by sector \(i\).
\item
  In a simple Economy, (value) Total labor demand is equal to total
  Household income. \(L_a + L_m + L_s = L = y_a + y_m + y_s\)
\end{itemize}

\subsection{}\label{section-1}

This simple Economy can be represented as follows: \[
\begin{aligned}
X_{11} &+ X_{12} &+ X_{13} &+ Y_{1}  &= X_{1} \\
X_{21} &+ X_{22} &+ X_{23} &+ Y_{2} &= X_{2} \\
X_{31} &+ X_{32} &+ X_{33} &+ Y_{3} &= X_{3} \\
L_1 &+ L_2 &+ L_3 &\ \ &= L 
\end{aligned}
\]

\subsection{}\label{section-2}

From the consumption/inputs Side, we could also write the equations as:

\[
\begin{aligned}
X_{11} &+ X_{21} &+ X_{31} &+ L_{1}  &= X_{1} \\
X_{12} &+ X_{22} &+ X_{32} &+ L_{2} &= X_{2} \\
X_{13} &+ X_{23} &+ X_{33} &+ L_{3} &= X_{3} \\
Y_1 &+ Y_2 &+ Y_3 &  &= Y 
\end{aligned}
\]

And from here we can get the technical coefficients: \[
\begin{aligned}
a_{11} X_1 &+ a_{21} X_1 &+ a_{31} X_1 &+ \lambda_{1} X_1 &= X_1 \\
a_{12} X_2 &+ a_{22} X_2 &+ a_{32} X_2 &+ \lambda_{2} X_2 &= X_2 \\
a_{13} X_3 &+ a_{23} X_3 &+ a_{33} X_3 &+ \lambda_{3} X_3 &= X_3 \\
\delta_1 Y &+ \delta_2 Y & + \delta_3 Y &  &= Y 
\end{aligned}
\]

Where \(a_{ij} = \frac{X_{ij}}{X_j}\) and
\(\lambda_i = \frac{L_i}{X_i}\)

\[a_{1i}+a_{2i}+a_{3i}+\lambda_i = 1 \ \& \ \delta_1+\delta_2+\delta_3 =1\]

\subsection{}\label{section-3}

With this, we can write the IO table

\[
\begin{aligned}
a_{11} X_1 &+ a_{12} X_2 &+ a_{13} X_3  &+ \delta_1 Y &= X_{1} \\
a_{21} X_1 &+ a_{22} X_2 &+ a_{23} X_3  &+ \delta_2 Y &= X_{2} \\
a_{31} X_1 &+ a_{32} X_2 &+ a_{33} X_3  &+ \delta_3 Y &= X_{3} \\
\lambda_1 X_1 &+ \lambda_2 X_2 &+ \lambda_3 X_3 &\ \ &= L 
\end{aligned} 
\]

Or into Matrix Form \[
\begin{pmatrix}
a_{11} & a_{12} & a_{13}  \\
a_{21} & a_{22} & a_{23} \\
a_{31} & a_{32} & a_{33} \\
\lambda_1 & \lambda_2 & \lambda_3 
\end{pmatrix}
\begin{pmatrix}
X_1 \\ X_2 \\ X_3 
\end{pmatrix} + 
\begin{pmatrix}
Y_1 \\ Y_2 \\ Y_3 \\ 0
\end{pmatrix}=
\begin{pmatrix}
X_1 \\ X_2 \\ X_3 \\ L
\end{pmatrix}
\]

\subsection{}\label{section-4}

Solve for Production sectors:

\[
\begin{pmatrix}
a_{11} & a_{12} & a_{13}  \\
a_{21} & a_{22} & a_{23} \\
a_{31} & a_{32} & a_{33} 
\end{pmatrix}
\begin{pmatrix}
X_1 \\ X_2 \\ X_3 
\end{pmatrix} + 
\begin{pmatrix}
Y_1 \\ Y_2 \\ Y_3 
\end{pmatrix}=
\begin{pmatrix}
X_1 \\ X_2 \\ X_3 
\end{pmatrix}
\]

\[
\begin{pmatrix}
Y_1 \\ Y_2 \\ Y_3 
\end{pmatrix}=I \begin{pmatrix} X_1 \\ X_2 \\ X_3 \end{pmatrix} -
A \begin{pmatrix} X_1 \\ X_2 \\ X_3 \end{pmatrix} = (I-A) \begin{pmatrix} X_1 \\ X_2 \\ X_3 \end{pmatrix}
\]

Finally:

\[
\begin{pmatrix} X_1 \\ X_2 \\ X_3 \end{pmatrix} = (I-A)^{-1} \begin{pmatrix} Y_1 \\ Y_2 \\ Y_3 \end{pmatrix}
\rightarrow 
\begin{pmatrix} \Delta X_1 \\ \Delta X_2 \\ \Delta X_3 \end{pmatrix} = (I-A)^{-1}
\begin{pmatrix} \Delta Y_1 \\ \Delta Y_2 \\ \Delta Y_3 \end{pmatrix}
\]

\subsection{Example}\label{example}

\begin{itemize}
\tightlist
\item
  Consider the following IO table for a simple economy with 3 sectors
  and a final consumer.
\end{itemize}

\begin{longtable}[]{@{}
  >{\raggedright\arraybackslash}p{(\columnwidth - 8\tabcolsep) * \real{0.2174}}
  >{\centering\arraybackslash}p{(\columnwidth - 8\tabcolsep) * \real{0.1884}}
  >{\centering\arraybackslash}p{(\columnwidth - 8\tabcolsep) * \real{0.2174}}
  >{\centering\arraybackslash}p{(\columnwidth - 8\tabcolsep) * \real{0.1449}}
  >{\centering\arraybackslash}p{(\columnwidth - 8\tabcolsep) * \real{0.2319}}@{}}
\toprule\noalign{}
\begin{minipage}[b]{\linewidth}\raggedright
Sector
\end{minipage} & \begin{minipage}[b]{\linewidth}\centering
Agriculture
\end{minipage} & \begin{minipage}[b]{\linewidth}\centering
Manufacturing
\end{minipage} & \begin{minipage}[b]{\linewidth}\centering
Services
\end{minipage} & \begin{minipage}[b]{\linewidth}\centering
Final Consumer
\end{minipage} \\
\midrule\noalign{}
\endhead
\bottomrule\noalign{}
\endlastfoot
Agriculture & 102 & 103 & 153 & 129 \\
Manufacturing & 133 & 124 & 77 & 99 \\
Services & 71 & 92 & 51 & 165 \\
Households & 181 & 114 & 98 & 0 \\
\end{longtable}

S1: What is the total production of each sector?

\begin{itemize}
\tightlist
\item
  \(X_1 = 102 + 103 + 153 + 129 = 487\)
\item
  \(X_2 = 133 + 124 + 77 + 99 = 433\)
\item
  \(X_3 = 71 + 92 + 51 + 165 = 379\)
\end{itemize}

S2: What are the technical coefficients?

\begin{longtable}[]{@{}
  >{\raggedright\arraybackslash}p{(\columnwidth - 8\tabcolsep) * \real{0.2027}}
  >{\centering\arraybackslash}p{(\columnwidth - 8\tabcolsep) * \real{0.2432}}
  >{\centering\arraybackslash}p{(\columnwidth - 8\tabcolsep) * \real{0.2027}}
  >{\centering\arraybackslash}p{(\columnwidth - 8\tabcolsep) * \real{0.1351}}
  >{\centering\arraybackslash}p{(\columnwidth - 8\tabcolsep) * \real{0.2162}}@{}}
\toprule\noalign{}
\begin{minipage}[b]{\linewidth}\raggedright
Sector
\end{minipage} & \begin{minipage}[b]{\linewidth}\centering
Agriculture
\end{minipage} & \begin{minipage}[b]{\linewidth}\centering
Manufacturing
\end{minipage} & \begin{minipage}[b]{\linewidth}\centering
Services
\end{minipage} & \begin{minipage}[b]{\linewidth}\centering
Final Consumer
\end{minipage} \\
\midrule\noalign{}
\endhead
\bottomrule\noalign{}
\endlastfoot
Agriculture & \(a_{11}\) = 0.209 & 0.238 & 0.404 & 0.328 \\
Manufacturing & 0.273 & 0.286 & 0.203 & 0.252 \\
Services & 0.146 & 0.212 & 0.135 & 0.420 \\
Households & 0.372 & 0.263 & 0.259 & 0.000 \\
\end{longtable}

This captures a snapshot of an economy. And could use to simulate
changes in production and labor demand.

S3. How much would production change if the final consumer demand for
Agriculture increases in 20\%?

\[\Delta X = (I-A)^-1
\begin{pmatrix} 0.2 * 129 \\ 0 \\ 0 \end{pmatrix} = 
\begin{pmatrix} 45.54 \\ 21.08 \\ 12.84 \end{pmatrix}
\]

S4: How much would labor demand change if the final consumer demand for
Agriculture increases in 20\%? \[\Delta L_i = \lambda_i * \Delta X_i\]

\(\Delta L_1 = 0.372 * 45.54 = 16.95 ; \Delta L_2 = 0.263 * 21.08 = 5.544 ; \Delta L_3 = 0.259 * 12.84 = 3.326\)

\subsection{\texorpdfstring{Example -
\texttt{Stata}}{Example - Stata}}\label{example---stata}

\texttt{mata}: Input Data

\begin{Shaded}
\begin{Highlighting}[]
\KeywordTok{mata}\NormalTok{: x  = (102, 103, 153 \textbackslash{} 133, 124, 77 \textbackslash{} 71, 92, 51)}
\KeywordTok{mata}\NormalTok{: }\FunctionTok{y}\NormalTok{  = (129 \textbackslash{} 99 \textbackslash{} 165)}
\KeywordTok{mata}\NormalTok{: hh = ( 181 , 114 , 98)}
\end{Highlighting}
\end{Shaded}

Estimate Total Production

\begin{Shaded}
\begin{Highlighting}[]
\KeywordTok{mata}\NormalTok{: tp = colsum(x):+hh ; tp}
\end{Highlighting}
\end{Shaded}

\begin{verbatim}
         1     2     3
    +-------------------+
  1 |  487   433   379  |
    +-------------------+
\end{verbatim}

Estimate Technical Coefficients

\begin{Shaded}
\begin{Highlighting}[]
\CommentTok{// Technical Coefficients}
\KeywordTok{mata}\NormalTok{:ai = x:/tp ; ai}
\end{Highlighting}
\end{Shaded}

\begin{verbatim}
                 1             2             3
    +-------------------------------------------+
  1 |  .2094455852   .2378752887   .4036939314  |
  2 |   .273100616   .2863741339   .2031662269  |
  3 |  .1457905544   .2124711316   .1345646438  |
    +-------------------------------------------+
\end{verbatim}

Estimate Change in Demand: 20\% increase in Agriculture

\begin{Shaded}
\begin{Highlighting}[]
\CommentTok{// Technical Coefficients}
\KeywordTok{mata}\NormalTok{:dy = }\FunctionTok{y}\NormalTok{ :* (.2 \textbackslash{} 0 \textbackslash{} 0); dy}
\end{Highlighting}
\end{Shaded}

\begin{verbatim}
          1
    +--------+
  1 |  25.8  |
  2 |     0  |
  3 |     0  |
    +--------+
\end{verbatim}

Estimate Change in Production

\begin{Shaded}
\begin{Highlighting}[]
\CommentTok{// Change in Production}
\KeywordTok{mata}\NormalTok{:dx = }\KeywordTok{qrinv}\NormalTok{(}\FunctionTok{I}\NormalTok{(3){-}ai)*dy; dx}
\end{Highlighting}
\end{Shaded}

\begin{verbatim}
                 1
    +---------------+
  1 |  45.54130481  |
  2 |  21.08637814  |
  3 |  12.84872246  |
    +---------------+
\end{verbatim}

Estimate Change in Labor Demand

\begin{Shaded}
\begin{Highlighting}[]
\KeywordTok{mata}\NormalTok{:dl = (hh:/tp)\textquotesingle{}:*dx; dl}
\end{Highlighting}
\end{Shaded}

\begin{verbatim}
                 1
    +---------------+
  1 |   16.9260291  |
  2 |  5.551609949  |
  3 |  3.322360954  |
    +---------------+
\end{verbatim}

\section{SAM: Social Accounting
Matrix}\label{sam-social-accounting-matrix}

\subsection{SAM: Social Accounting
Matrix}\label{sam-social-accounting-matrix-1}

\begin{itemize}
\item
  SAM can be thought as an upgraded version of IO-tables.
\item
  They are a way to organize information about the production structure
  of an economy, but also the distribution of resources.

  \begin{itemize}
  \tightlist
  \item
    This it will not only register production of goods and services, but
    also transfers of resources between sectors and agents.
  \end{itemize}
\item
  You can also use it as basis for a plausible model of the economy.

  \begin{itemize}
  \tightlist
  \item
    Prediction of changes in production, income and distribution.
  \end{itemize}
\end{itemize}

\subsection{Example}\label{example-1}

\begin{longtable}[]{@{}lcccc@{}}
\caption{Closed Economy No GoV SAM}\tabularnewline
\toprule\noalign{}
& Production & Consumption & Accumulation & Totals \\
\midrule\noalign{}
\endfirsthead
\toprule\noalign{}
& Production & Consumption & Accumulation & Totals \\
\midrule\noalign{}
\endhead
\bottomrule\noalign{}
\endlastfoot
Production & & \(C\) & \(I\) & \(C+I\) \\
Consumption & \(Y\) & & & \(Y\) \\
Accumulation & & \(S\) & & \(S\) \\
Totals & \(Y\) & \(C+S\) & \(I\) & \\
\end{longtable}

\begin{itemize}
\tightlist
\item
  Goods/services are Transfered from left to Top-right
\item
  Monetary Transfers are from Top-right to left
\end{itemize}

\subsection{Example 2}\label{example-2}

\begin{longtable}[]{@{}
  >{\raggedright\arraybackslash}p{(\columnwidth - 12\tabcolsep) * \real{0.1618}}
  >{\centering\arraybackslash}p{(\columnwidth - 12\tabcolsep) * \real{0.1029}}
  >{\centering\arraybackslash}p{(\columnwidth - 12\tabcolsep) * \real{0.1765}}
  >{\centering\arraybackslash}p{(\columnwidth - 12\tabcolsep) * \real{0.1324}}
  >{\centering\arraybackslash}p{(\columnwidth - 12\tabcolsep) * \real{0.0735}}
  >{\centering\arraybackslash}p{(\columnwidth - 12\tabcolsep) * \real{0.1324}}
  >{\centering\arraybackslash}p{(\columnwidth - 12\tabcolsep) * \real{0.2206}}@{}}
\caption{Open Economy with GoV SAM}\tabularnewline
\toprule\noalign{}
\begin{minipage}[b]{\linewidth}\raggedright
\end{minipage} & \begin{minipage}[b]{\linewidth}\centering
S1
\end{minipage} & \begin{minipage}[b]{\linewidth}\centering
S2
\end{minipage} & \begin{minipage}[b]{\linewidth}\centering
S3
\end{minipage} & \begin{minipage}[b]{\linewidth}\centering
S4
\end{minipage} & \begin{minipage}[b]{\linewidth}\centering
S5
\end{minipage} & \begin{minipage}[b]{\linewidth}\centering
Totals
\end{minipage} \\
\midrule\noalign{}
\endfirsthead
\toprule\noalign{}
\begin{minipage}[b]{\linewidth}\raggedright
\end{minipage} & \begin{minipage}[b]{\linewidth}\centering
S1
\end{minipage} & \begin{minipage}[b]{\linewidth}\centering
S2
\end{minipage} & \begin{minipage}[b]{\linewidth}\centering
S3
\end{minipage} & \begin{minipage}[b]{\linewidth}\centering
S4
\end{minipage} & \begin{minipage}[b]{\linewidth}\centering
S5
\end{minipage} & \begin{minipage}[b]{\linewidth}\centering
Totals
\end{minipage} \\
\midrule\noalign{}
\endhead
\bottomrule\noalign{}
\endlastfoot
S1: Prod & & \(C\) & \(G\) & \(I\) & \(E\) & \(C+G+I+E\) \\
S2: HH & \(Y\) & & & & & \(Y\) \\
S3: Gov & & \(Tx\) & & & & \(Tx\) \\
S4: K acc & & \(S_h\) & \(S_g\) & & \(S_f\) & \(S_h+S_g+S_f\) \\
S5: RofW & \(M\) & & & & & \(M\) \\
Totals & \(Y+M\) & \(C+S_h+Tx\) & \(G+S_g\) & \(I\) & \(E+S_f\) & \\
\end{longtable}

Here \(S1\) and \(S2\) is what we had in the IO table. Thus, we could
further expand the SAM to include more sectors and agents.

\subsection{Example 3}\label{example-3}

\begin{longtable}[]{@{}
  >{\raggedright\arraybackslash}p{(\columnwidth - 16\tabcolsep) * \real{0.1667}}
  >{\centering\arraybackslash}p{(\columnwidth - 16\tabcolsep) * \real{0.1282}}
  >{\centering\arraybackslash}p{(\columnwidth - 16\tabcolsep) * \real{0.0897}}
  >{\centering\arraybackslash}p{(\columnwidth - 16\tabcolsep) * \real{0.1026}}
  >{\centering\arraybackslash}p{(\columnwidth - 16\tabcolsep) * \real{0.1154}}
  >{\centering\arraybackslash}p{(\columnwidth - 16\tabcolsep) * \real{0.1026}}
  >{\centering\arraybackslash}p{(\columnwidth - 16\tabcolsep) * \real{0.0897}}
  >{\centering\arraybackslash}p{(\columnwidth - 16\tabcolsep) * \real{0.1026}}
  >{\centering\arraybackslash}p{(\columnwidth - 16\tabcolsep) * \real{0.1026}}@{}}
\caption{MoreDetailed SAM}\tabularnewline
\toprule\noalign{}
\begin{minipage}[b]{\linewidth}\raggedright
\end{minipage} & \begin{minipage}[b]{\linewidth}\centering
S1
\end{minipage} & \begin{minipage}[b]{\linewidth}\centering
S2
\end{minipage} & \begin{minipage}[b]{\linewidth}\centering
S3
\end{minipage} & \begin{minipage}[b]{\linewidth}\centering
S4
\end{minipage} & \begin{minipage}[b]{\linewidth}\centering
S5
\end{minipage} & \begin{minipage}[b]{\linewidth}\centering
S6
\end{minipage} & \begin{minipage}[b]{\linewidth}\centering
S7
\end{minipage} & \begin{minipage}[b]{\linewidth}\centering
S8
\end{minipage} \\
\midrule\noalign{}
\endfirsthead
\toprule\noalign{}
\begin{minipage}[b]{\linewidth}\raggedright
\end{minipage} & \begin{minipage}[b]{\linewidth}\centering
S1
\end{minipage} & \begin{minipage}[b]{\linewidth}\centering
S2
\end{minipage} & \begin{minipage}[b]{\linewidth}\centering
S3
\end{minipage} & \begin{minipage}[b]{\linewidth}\centering
S4
\end{minipage} & \begin{minipage}[b]{\linewidth}\centering
S5
\end{minipage} & \begin{minipage}[b]{\linewidth}\centering
S6
\end{minipage} & \begin{minipage}[b]{\linewidth}\centering
S7
\end{minipage} & \begin{minipage}[b]{\linewidth}\centering
S8
\end{minipage} \\
\midrule\noalign{}
\endhead
\bottomrule\noalign{}
\endlastfoot
S1: Act & & \(Gds\) & & & & & & \\
S2: Commod & \(IntGds\) & & & & \(C\) & \(G\) & \(I\) & \(E\) \\
S3: Factors & \(VA\) & & & & & & & \(FE\) \\
S4: Enter & & & \(Prof\) & & & \(ETr\) & & \\
S5: HH & & & \(Wage\) & \(DProf\) & & \(Tr\) & & \(REM\) \\
S6: Gov & \(ITx\) & & \(FTx\) & \(ETx\) & \(DTx\) & & & \(Tarf\) \\
S7: K acc & & & & \(RetY\) & \(S_h\) & \(S_g\) & & \(KTrM\) \\
S8: RofW & & \(M\) & \(FM\) & & \(REMA\) & \(TrA\) & \(KtrA\) & \\
\end{longtable}

\subsection{Other Extensions}\label{other-extensions}

\begin{itemize}
\tightlist
\item
  SAM also allow you to do further extensions to include more agents
  (heterogenous)

  \begin{itemize}
  \tightlist
  \item
    Green-Industry
  \item
    Informal Sector
  \item
    Households by income level
  \item
    etc.
  \end{itemize}
\end{itemize}

\section{Thats all folks!}\label{thats-all-folks}

\subsection{What to get from today?}\label{what-to-get-from-today}

\begin{itemize}
\tightlist
\item
  How to use weights in Stata to account for survey design, and how to
  obtain summary statistics.
\item
  How to test differences in means across groups.
\item
  How to use IO tables to simulate changes in production and labor
  demand.
\item
  Understand how SAM can be used to represent an Economy
\end{itemize}



\end{document}
